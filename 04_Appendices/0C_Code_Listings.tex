\chapter{Including Code Listings}\label{app:codelistings}
	This appendix provides guidelines for including code listings in your thesis. 
	Code listings are often used to demonstrate algorithms, data processing scripts, or other relevant programming content. 
	Proper formatting ensures that code is both readable and aesthetically pleasing.

	\section{Using the \pkg{listings} Package}
		The \pkg{listings} package is a powerful tool for displaying code in LaTeX. 
		It supports syntax highlighting for a wide variety of programming languages and offers many customization options.

		\subsection{Basic Usage}

			To include a simple code listing, you can use the following command:

\noindent\lstinline|\begin{lstlisting}[language=Python]|\\[-0.7em]
\lstinline|	# Your code here|\\[-0.7em]
\lstinline|	print("Hello, world!")|\\[-0.7em]
\lstinline|\end{lstlisting}|

			The \opt{language} option specifies the programming language, which enables syntax highlighting. 
			Replace \opt{Python} with the appropriate language for your code.

			% INSERT EXAMPLE: Consider adding a few more examples showing code listings in different languages (e.g., C++, Java, MATLAB).

		\subsection{Customizing Listings}

			The \pkg{listings} package allows for extensive customization. 
			You can adjust the appearance of your code by setting options such as \opt{frame}, \opt{backgroundcolor}, \opt{keywordstyle}, and more.

			Here is an example of how to customize your code listing:

			\begin{lstlisting}[style=LaTeXStyle]
\lstset{
	language=Python,
	frame=single,
	backgroundcolor=\color{gray!10},
	keywordstyle=\color{blue}\bfseries,
	commentstyle=\color{green},
	stringstyle=\color{red},
	basicstyle=\ttfamily,
	breaklines=true
}
			\end{lstlisting}

			This configuration adds a single-line frame around the code, sets a light gray background, and defines styles for keywords, comments, and strings.
			
			\lstinputlisting[caption=This is a caption for the inserted code,style=MatlabStyle]{\insertcode{matlabCode.m}}
			
			\lstinputlisting[caption=This is a caption for the inserted code,style=CStyle]{\insertcode{cCode.cpp}}
			


			% INSERT EXAMPLE: Include an example of a code listing using this or a similar custom configuration.

	\section{Advanced Features}

		\subsection{Including External Files}

			The \texttt{listings} package allows you to include code from external files. 
			This is particularly useful if you have long code files that you want to reference directly.

			\begin{lstlisting}[style=LaTeXStyle]
\lstinputlisting[language=Python]{path/to/your/code.py}
			\end{lstlisting}

			Replace \opt{path/to/your/code.py} with the actual path to your file.
			Or you can use the \cmd{addmedia}\mopt{./99_Inclusions/} and \cmd{addcode}\mopt{Code/} commands to define the location for code files, then you can just use \cmd{insertcode}\mopt{filename.ext} command instead of the full path.
			You can customize the display in the same way as inline listings.

			% INSERT EXAMPLE: Demonstrate including an external file and discuss any specific considerations, such as relative vs. absolute paths.

		\subsection{Handling Special Characters}
			If your code contains special characters (\textit{e.g.}, \texttt{\#}, \texttt{\%}, \texttt{\$}), you may need to escape them or use the \texttt{literate} option to ensure proper display.

			\begin{lstlisting}[style=LaTeXStyle]
\lstset{
	literate={~} {$\sim$}{1}
}
			\end{lstlisting}

			This command, for example, replaces the tilde symbol with the appropriate LaTeX command.

			% INSERT EXAMPLE: Include an example where special characters are handled correctly.

	\section{Line Breaks in Long Code Lines}

		To automatically break long lines of code, use the \opt{breaklines=true} option as shown in the earlier examples. 
		This prevents code from running off the page and maintains readability.

			% INSERT EXAMPLE: Provide a code snippet with long lines that demonstrate automatic line breaks.
	\section{Conclusion}

		Including well-formatted code listings in your thesis can enhance the clarity of your work and demonstrate your technical skills. 
		By following the guidelines in this appendix, you can ensure that your code is presented professionally.

		% INSERT FINAL NOTES: Consider adding any university-specific guidelines or final remarks relevant to code listings.