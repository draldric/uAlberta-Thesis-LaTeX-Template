\chapter{Additional Example Figures}\label{app:examplefigures}
	Each of the following pages will provide an example of a different figure configuration. 
	In addition to the examples the code that generates the figure will be provided and explanations of what the different parts of the code do will be included.
	From all of the included information in this \nameCref{app:examplefigures} it should be possible to even develop your own figures that potentially suit your needs best.
		
		\clearpage



		\vspace*{\fill}
		\begin{figure}[H]
			\centering
			\begin{subfigure}{0.45\linewidth}
				\includegraphics[width=\linewidth]{example-image}
				\caption{} % Leave blank for just letter
				\label{fig:doubleImage2:a}
			\end{subfigure}
			~
			\begin{subfigure}{0.45\linewidth}
				\includegraphics[width=\linewidth]{example-image}
				\caption{} % Leave blank for just letter
				\label{fig:doubleImage2:b}
			\end{subfigure}
			\caption{This is an example of a double image figure.}
			\label{fig:doubleImage2}
		\end{figure}
		\begin{lstlisting}[float=ht,caption=,label=lst:doubleImage,style=LaTeXStyle,basicstyle=\tiny\ttfamily,]
\begin{figure}[H]
	\centering
	\begin{subfigure}{0.45\linewidth}
		\includegraphics[width=\linewidth]{example-image}
		\caption{} % Leave blank for just letter
		\label{fig:doubleImage2:a}
	\end{subfigure}
	~
	\begin{subfigure}{0.45\linewidth}
		\includegraphics[width=\linewidth]{example-image}
		\caption{} % Leave blank for just letter
		\label{fig:doubleImage2:b}
	\end{subfigure}
	\caption{This is an example of a double image figure.}
	\label{fig:doubleImage2}
\end{figure}
		\end{lstlisting}
		\vspace*{\fill}
		\pagebreak



		\vspace*{\fill}
		\begin{figure}[H]
			\centering
			\hspace*{\fill}% Adds space to left of top image (prevents two images from going to top)
			\begin{subfigure}{0.45\linewidth}
				\includegraphics[width=\linewidth]{example-image}
				\caption{} % Leave blank for just letter
				\label{fig:tripleImage1:a}
			\end{subfigure}
			\hspace*{\fill} % Adds space to right of top image (prevents two images from going to top)
			\par\vspace{1em}% Adds space between upper and lower images
			\begin{subfigure}{0.45\linewidth}
				\includegraphics[width=\linewidth]{example-image}
				\caption{} % Leave blank for just letter
				\label{fig:tripleImage1:b}
			\end{subfigure}
			~ % Adds space between the two lower figures
			\begin{subfigure}{0.45\linewidth}
				\includegraphics[width=\linewidth]{example-image}
				\caption{} % Leave blank for just letter
				\label{fig:tripleImage1:c}
			\end{subfigure}
			\caption{This is an example of a triple image figure.}
			\label{fig:tripleImage1}
		\end{figure}
		\begin{lstlisting}[float=ht,caption=,label=lst:tripleImage1,style=LaTeXStyle,basicstyle=\tiny\ttfamily,]
\begin{figure}[H]
	\centering
	\hspace*{\fill}% Adds space to left of top image (prevents two images from going to top)
	\begin{subfigure}{0.45\linewidth}
		\includegraphics[width=\linewidth]{example-image}
		\caption{} % Leave blank for just letter
		\label{fig:tripleImage1:a}
	\end{subfigure}
	\hspace*{\fill} % Adds space to right of top image (prevents two images from going to top)
	\par\vspace{1em}% Adds space between upper and lower images
	\begin{subfigure}{0.45\linewidth}
		\includegraphics[width=\linewidth]{example-image}
		\caption{} % Leave blank for just letter
		\label{fig:tripleImage1:b}
	\end{subfigure}
	~ % Adds space between the two lower figures
	\begin{subfigure}{0.45\linewidth}
		\includegraphics[width=\linewidth]{example-image}
		\caption{} % Leave blank for just letter
		\label{fig:tripleImage1:c}
	\end{subfigure}
	\caption{This is an example of a triple image figure.}
	\label{fig:tripleImage1}
\end{figure}
		\end{lstlisting}
		\vspace*{\fill}
		\pagebreak



		\vspace*{\fill}
		\begin{figure}[H]
			\centering
			\hspace*{\fill}% Adds space to left of top image (prevents two images from going to top)
			\begin{subfigure}{0.60\linewidth+1em} % 0.9 = 0.45 + 0.45, and 1em is the width of ~
				\includegraphics[width=\linewidth]{example-image}
				\caption{} % Leave blank for just letter
				\label{fig:tripleImage2:a}
			\end{subfigure}
			\hspace*{\fill} % Adds space to right of top image (prevents two images from going to top)
			\par\vspace{1em}% Adds space between upper and lower images
			\begin{subfigure}{0.30\linewidth}
				\includegraphics[width=\linewidth]{example-image}
				\caption{} % Leave blank for just letter
				\label{fig:tripleImage2:b}
			\end{subfigure}
			~ % Adds space between the two lower figures
			\begin{subfigure}{0.30\linewidth}
				\includegraphics[width=\linewidth]{example-image}
				\caption{} % Leave blank for just letter
				\label{fig:tripleImage2:c}
			\end{subfigure}
			\caption{This is a second example of a triple image figure.}
			\label{fig:tripleImage2}
		\end{figure}
		\begin{lstlisting}[float=ht,caption=,label=lst:tripleImage2,style=LaTeXStyle,basicstyle=\tiny\ttfamily,]
\begin{figure}[H]
	\centering
	\hspace*{\fill}% Adds space to left of top image (prevents two images from going to top)
	\begin{subfigure}{0.90\linewidth+1em} % 0.9 = 0.45 + 0.45, and 1em is the width of ~
		\includegraphics[width=\linewidth]{example-image}
		\caption{} % Leave blank for just letter
		\label{fig:tripleImage2:a}
	\end{subfigure}
	\hspace*{\fill} % Adds space to right of top image (prevents two images from going to top)
	\par\vspace{1em}% Adds space between upper and lower images
	\begin{subfigure}{0.45\linewidth}
		\includegraphics[width=\linewidth]{example-image}
		\caption{} % Leave blank for just letter
		\label{fig:tripleImage2:b}
	\end{subfigure}
	~ % Adds space between the two lower figures
	\begin{subfigure}{0.45\linewidth}
		\includegraphics[width=\linewidth]{example-image}
		\caption{} % Leave blank for just letter
		\label{fig:tripleImage2:c}
	\end{subfigure}
	\caption{This is a second example of a triple image figure.}
	\label{fig:tripleImage2}
\end{figure}
		\end{lstlisting}
		\vspace*{\fill}
		\pagebreak



		\vspace*{\fill}
		\begin{figure}[H]
			\centering
			\begin{subfigure}{0.45\linewidth}
				\includegraphics[width=\linewidth]{example-image}
				\caption{} % Leave blank for just letter
				\label{fig:quadImage:a}
			\end{subfigure}
			~ % Adds space between the two top figures
			\begin{subfigure}{0.45\linewidth}
				\includegraphics[width=\linewidth]{example-image}
				\caption{} % Leave blank for just letter
				\label{fig:quadImage:b}
			\end{subfigure}
			\par\vspace{1em} % Adds space between upper and lower images
			\begin{subfigure}{0.45\linewidth}
				\includegraphics[width=\linewidth]{example-image}
				\caption{} % Leave blank for just letter
				\label{fig:quadImage:c}
			\end{subfigure}
			~ % Adds space between the two lower figures
			\begin{subfigure}{0.45\linewidth}
				\includegraphics[width=\linewidth]{example-image}
				\caption{} % Leave blank for just letter
				\label{fig:quadImage:d}
			\end{subfigure}
			\caption{This is an example of a quad image figure.}
			\label{fig:quadImage}
		\end{figure}
		\begin{lstlisting}[float=ht,caption=,label=lst:quadImage,style=LaTeXStyle,basicstyle=\tiny\ttfamily,]
\begin{figure}[H]
	\centering
	\begin{subfigure}{0.45\linewidth}
		\includegraphics[width=\linewidth]{example-image}
		\caption{} % Leave blank for just letter
		\label{fig:quadImage:a}
	\end{subfigure}
	~ % Adds space between the two top figures
	\begin{subfigure}{0.45\linewidth}
		\includegraphics[width=\linewidth]{example-image}
		\caption{} % Leave blank for just letter
		\label{fig:quadImage:b}
	\end{subfigure}
	\par\vspace{1em} % Adds space between upper and lower images
	\begin{subfigure}{0.45\linewidth}
		\includegraphics[width=\linewidth]{example-image}
		\caption{} % Leave blank for just letter
		\label{fig:quadImage:c}
	\end{subfigure}
	~ % Adds space between the two lower figures
	\begin{subfigure}{0.45\linewidth}
		\includegraphics[width=\linewidth]{example-image}
		\caption{} % Leave blank for just letter
		\label{fig:quadImage:d}
	\end{subfigure}
	\caption{This is an example of a quad image figure.}
	\label{fig:quadImage}
\end{figure}
		\end{lstlisting}
		\vspace*{\fill}
		\pagebreak