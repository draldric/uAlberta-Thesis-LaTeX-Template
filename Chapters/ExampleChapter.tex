\chapter{Example Chapter}\label{ch:Example}
  This chapter aims to provide examples how how to structure and create specific components in your thesis document. The very first one is showing a citation, like the one at the end of this sentence \cite{TEST}. The second shows how to create more than one citation and how they are grouped \cite{testone,cite2,cite3,cite4,cite5}.
This sentence shows how a gap in the citations is handled \cite{testone,cite2,cite3,cite5}. 
  \section{Tables}
	
  % X - Justified (Equal Spacing)
  % L - Left Aligned (Equal Spacing)
  % C - Center Aligned (Equal Spacing)
  % R - Right Aligned (Equal Spacing)
  % l - Left Aligned (Fit to Contents)
  % c - Center Aligned (Fit to Contents)
  % r - Right Aligned (Fit to Contents)
  
  \begin{table}[!htb]
    \caption{This is a basic table}
    \centering
    \begin{tabularx}{0.75\textwidth}{LCR} 
      % Equally spaced cells that are left, center, and reight aligned. 
      % The entire table will be 75% the width of the text.
      \hline
      \textbf{Left Aligned Title} & \textbf{Centered Title} & \textbf{Right Aligned Title} \\\hline
      This is left aligned & This is centered & This is right aligned \\
      This is left aligned & This is centered & This is right aligned \\
      This is left aligned & This is centered & This is right aligned \\
      This is left aligned & This is centered & This is right aligned \\\hline
    \end{tabularx}
    \label{tab:basicTable}
  \end{table}
  
  \begin{table}[!htb]
    \caption{This is a complex table.}
    \centering
    \begin{tabularx}{\textwidth}{lCR}
      % Left most cell is fitted to the content.
      % The center and right columns are equally spaced cells that are center, and reight aligned. 
      % The entire table will be 75% the width of the text.
      \hline
      \multirow{2}{*}{\textbf{This is two row\quad}} & \multicolumn{2}{c}{\textbf{This is two columns}}\\\cline{2-3} % \cline draws a partial line across cells #-#
       & \textbf{Centered Title} & \textbf{Right Aligned Title} \\\hline
      \multirow{2}{*}{This is two row} & This is centered & This is right aligned \\
       & This is centered & This is right aligned \\\cline{1-1}
      \multirow{2}{*}{This is two row} & This is centered & This is right aligned \\
       & This is centered & This is right aligned \\\hline
    \end{tabularx}
    \label{tab:complexTable}
  \end{table}
  
  
  \section{Figures}
  This section will provide examples of how to create figures, and different types of multi/sub-figures. Additionally, if you have many figures in a section and they are bleeding too much into the following sections a \textbackslash{}clearpage command can be issued before the next section. However, note that this will force the next section to begin on a new page. Note that the first ``figure'' is actually a plate; a plate is the proper title associated with a photograph, using the environment `plate' instead of figure and command listofplates will generate everything for you.
  \begin{plate}[!htb]
    \centering
    \includegraphics[width=0.7\textwidth]{example-image}
    \caption{This is an example of a single image plate.}
    \label{fig:singleImage}
  \end{plate}
  
  \begin{figure}[!htb]
    \centering
    \begin{subfigure}{0.45\textwidth}
      \includegraphics[width=\textwidth]{example-image}
      \caption{} % Leave blank for just letter
      \label{fig:doubleImage:a}
    \end{subfigure}
    ~
    \begin{subfigure}{0.45\textwidth}
      \includegraphics[width=\textwidth]{example-image}
      \caption{} % Leave blank for just letter
      \label{fig:doubleImage:b}
    \end{subfigure}
    \caption{This is an example of a double image figure.}
    \label{fig:doubleImage}
  \end{figure}
  
  \begin{figure}[!htb]
    \centering
    \hspace*{\fill}% Adds space to left of top image (prevents two images from going to top)
    \begin{subfigure}{0.45\textwidth}
      \includegraphics[width=\textwidth]{example-image}
      \caption{} % Leave blank for just letter
      \label{fig:tripleImage:a}
    \end{subfigure}
    \hspace*{\fill} % Adds space to right of top image (prevents two images from going to top)
    \par\vspace{1em}% Adds space between upper and lower images
    \begin{subfigure}{0.45\textwidth}
      \includegraphics[width=\textwidth]{example-image}
      \caption{} % Leave blank for just letter
      \label{fig:tripleImage:b}
    \end{subfigure}
    ~ % Adds space between the two lower figures
    \begin{subfigure}{0.45\textwidth}
      \includegraphics[width=\textwidth]{example-image}
      \caption{} % Leave blank for just letter
      \label{fig:tripleImage:c}
    \end{subfigure}
    \caption{This is an example of a triple image figure.}
    \label{fig:tripleImage}
  \end{figure}
  
  \begin{figure}[!htb]
    \centering
    \hspace*{\fill}% Adds space to left of top image (prevents two images from going to top)
    \begin{subfigure}{0.90\textwidth+1em} % 0.9 = 0.45 + 0.45, and 1em is the width of ~
      \includegraphics[width=\textwidth]{example-image}
      \caption{} % Leave blank for just letter
      \label{fig:tripleImage:a}
    \end{subfigure}
    \hspace*{\fill} % Adds space to right of top image (prevents two images from going to top)
    \par\vspace{1em}% Adds space between upper and lower images
    \begin{subfigure}{0.45\textwidth}
      \includegraphics[width=\textwidth]{example-image}
      \caption{} % Leave blank for just letter
      \label{fig:tripleImage:b}
    \end{subfigure}
    ~ % Adds space between the two lower figures
    \begin{subfigure}{0.45\textwidth}
      \includegraphics[width=\textwidth]{example-image}
      \caption{} % Leave blank for just letter
      \label{fig:tripleImage:c}
    \end{subfigure}
    \caption{This is a second example of a triple image figure.}
    \label{fig:tripleImage}
  \end{figure}
  
  \begin{figure}[!htb]
    \centering
    \begin{subfigure}{0.45\textwidth}
      \includegraphics[width=\textwidth]{example-image}
      \caption{} % Leave blank for just letter
      \label{fig:quadImage:a}
    \end{subfigure}
    ~ % Adds space between the two top figures
    \begin{subfigure}{0.45\textwidth}
      \includegraphics[width=\textwidth]{example-image}
      \caption{} % Leave blank for just letter
      \label{fig:quadImage:b}
    \end{subfigure}
    \par\vspace{1em} % Adds space between upper and lower images
    \begin{subfigure}{0.45\textwidth}
      \includegraphics[width=\textwidth]{example-image}
      \caption{} % Leave blank for just letter
      \label{fig:quadImage:c}
    \end{subfigure}
    ~ % Adds space between the two lower figures
    \begin{subfigure}{0.45\textwidth}
      \includegraphics[width=\textwidth]{example-image}
      \caption{} % Leave blank for just letter
      \label{fig:quadImage:d}
    \end{subfigure}
    \caption{This is an example of a quad image figure.}
    \label{fig:quadImage}
  \end{figure}
  
  \clearpage % forces the remaining images (floats to be placed)
  \section{Equations}
  The following equation has no referencing number:
  \nonumeq{E & = m\ c^2}
  
  \Cref{eq:quickEq} has a reference to it though. Or for more control the source for \Cref{eq:quickEq} can be written out fully as it was for \Cref{eq:quickEq2}.
  
  \numeq{pi & = 3.1415...}{eq:quickEq} % shorthand for the following way of writing equations.
  \begin{align}\label{eq:quickEq2}
    e & = 2.7183...
  \end{align}
  
  If you have multiple equations that you want arranged very neatly, use the align environment and you can assign individual equations numbers as shown in \Cref{eq:multiref:a,eq:multiref:b,eq:multiref:c}.
  \begin{align}%Note: Alignment happens at the "=" character
    \label{eq:multiref:a} Equation1 & = 1 + 1\\
    \label{eq:multiref:b} Equation2 & = 2 + 2\\
    \label{eq:multiref:c} Equation3 & = 3 + 3
  \end{align}
  
  
  
  \printreferences % Add a Reference Section to the end of the Chapter.