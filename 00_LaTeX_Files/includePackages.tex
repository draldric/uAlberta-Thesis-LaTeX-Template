
\usepackage{nomencl}
    \makenomenclature
    \renewcommand\nomgroup[1]{%
        \item[\bfseries
        \ifstrequal{#1}{A}{Constants}{%
        \ifstrequal{#1}{B}{Latin}{%
        \ifstrequal{#1}{C}{Greek}{}}}]%
    }
    \renewcommand{\nomname}{List of Symbols}
    \newcommand{\nomunit}[1]{%
    \renewcommand{\nomentryend}{\hspace*{\fill}#1}}

\usepackage[acronym,nonumberlist]{glossaries}
    \makenoidxglossaries
    \newcommand{\addacronym}[2]{\newacronym{#1}{#1}{#2}}
    \newcommand{\addterm}[2]{\newglossaryentry{#1}{name={#1},description={#2}}}
    \newcommand\generateglossary{\pagestyle{plain}\thispagestyle{plain}\printnoidxglossary[title={\glossaryname}]}
    \newcommand\abbreviations{\pagestyle{plain}\thispagestyle{plain}\printnoidxglossary[type=\acronymtype,title={\abbreviationsname}]}

\usepackage[utf8]{inputenc}
\usepackage{comment}

% The basics for using math in LaTeX
	\usepackage{array}
	\usepackage{amsmath}
	\usepackage{amsfonts}
	\usepackage{amssymb}

% Used to provide underlining that can be broken across multiple lines
	\usepackage[normalem]{ulem} 

% Used for creating subfigures
	\usepackage{subcaption}

% The defacto package for including graphics
	\usepackage{graphicx}

% Used to extend or replace the table/tabular environment
	\usepackage{tabularx}  % Auto-Sizing Tables
	\usepackage{longtable} % Multi-page Tables
	\usepackage{booktabs}  % Nicer Lines
	\usepackage{multicol}  % Column Merging
	\usepackage{multirow}  % Row Merging
	\usepackage{makecell}  % Allows more control over cell contents

% Used for creating consistent plots and graphs
	\usepackage{pgfplots}

% 
	\usepackage{pdfpages}

% Used for inserting syntax highlighted code
	\usepackage{listings}

% Used for extending the rules/control for hyphenation
% \hyp allows the hyphenation of the words creating a compound word: electromagnetic\hyp{}endioscopy.
	\usepackage{hyphenat}

% Used for better quoting and formatting commands
	\usepackage{csquotes}
	\usepackage{ragged2e}

% Used for including hyperlinks, document links, and better cross-references
	\usepackage{hyperref}
		\hypersetup{
			colorlinks,
			linkcolor={red!50!black},
			citecolor={blue!50!black},
			urlcolor={blue!80!black},
			pdfencoding=unicode,
			breaklinks=true
		}
		\urlstyle{same}
		\Urlmuskip=0mu plus 1mu\relax
	\usepackage[nameinlink]{cleveref}

% Used for creating the Bibliography
%   Currently set up for IEEE style citations, with compact number in-text [1 - 3] vs [1, 2, 3]
%   To change the style to APA, MLA, or other change the style=ieee, and remove the citestyle=numeric-comp
	\usepackage[
		style=ieee,
		citestyle=numeric-comp,
		dashed=false,
		backend=bibtex,
		refsegment=chapter,
		sorting=none,
		defernumbers=true]{biblatex}

% USED ONLY FOR FORMATTING EXAMPLE DOCUMENT (DO NOT USE IN THESIS)
	\usepackage{float}

	\usepackage{fontawesome}
	\usepackage{lipsum}
	\usepackage{linegoal}
	\usepackage{xcolor}
%% COLOUR DEFINITIONS
    \definecolor{grey}{rgb}{0.3,0.3,0.3}
    \definecolor{commentGreen}{rgb}{0,0.8,0}
    \definecolor{codeGrey}{rgb}{0.9,0.9,0.9}
    \definecolor{R}{rgb}{0.9,0,0}
    \definecolor{Y}{rgb}{0.8,0.8,0}
    \definecolor{G}{rgb}{0,0.75,0}
    \definecolor{gold}{rgb}{0.4,0.4,0}
  
  
  %% NEW COMMANDS
    
	\newcommand{\exampleText}{\lipsum[1][1-4]}
  
    \newcommand{\commenting}   [1]{\noindent\textcolor{grey}{\textbf{\faCommenting{}} \parbox[t]{\linegoal}{\textit{#1}}}}
    \newcommand{\danger}       [1]{\noindent\textsc{\color{black}\colorbox{red}{\textbf{{\faWarning} DANGER}}} \parbox[t]{\linegoal}{\textbf{\MakeUppercase{#1}}}}
    \newcommand{\warning}      [1]{\noindent\textsc{\color{black}\colorbox{orange}{{\textbf{{\faWarning} WARNING}}}} \parbox[t]{\linegoal}{\textbf{#1}}}
    \newcommand{\caution}      [1]{\noindent\textsc{\color{black}\colorbox{yellow}{{\textbf{{\faWarning} CAUTION}}}} \parbox[t]{\linegoal}{#1}}
    \newcommand{\notice}       [1]{\noindent\textsc{\color{white}\colorbox{black}{{\textbf{\textit{NOTICE}}}}} \parbox[t]{\linegoal}{#1}}
    \newcommand{\alert}        [1]{\noindent\textbf{\faWarning} {\parbox[t]{\linegoal}{#1}}}
    \newcommand{\note}         [1]{\noindent\textbf{\textit{Note: }}\parbox[t]{\linegoal}{\textit{#1}}}
    \newcommand{\info}         [1]{\noindent\textcolor{blue}{\textbf{{\faInfoCircle{}}} \parbox[t]{\linegoal}{#1}}}
    \newcommand{\question}     [1]{\noindent\textcolor{orange}{{\textbf{{\faQuestionCircle}}}} \parbox[t]{\linegoal}{\textbf{#1}}}
    \newcommand{\answer}       [1]{\noindent\-\hspace{4ex}{\textsc{\color{orange}{{\textbf{{\faCheckCircle}}}}}} \parbox[t]{\linegoal}{\textit{#1}}}
    \newcommand{\qna}          [2]{\noindent\textcolor{orange}{{\textbf{{\faQuestionCircle}}}} \textbf{#1}\\\-\hspace{4ex} \parbox[t]{\linegoal}{{\textit{#2}}}}
    \newcommand{\tip}          [1]{\noindent\textcolor{gold}{{\textbf{{\faLightbulbO{}}}} \parbox[t]{\linegoal}{#1}}}
    \newcommand{\download}     [2]{\noindent\textcolor{black}{{\textbf{{\faDownload}}}} \parbox[t]{\linegoal}{\textattachfile[color=0 0 0]{#1}{#2}}}
    \newcommand{\clouddownload}[2]{\noindent\textbf{\faCloudDownload} \parbox[t]{\linegoal}{\href{#1}{#2}}}
    \newcommand{\important}    [1]{\noindent\textcolor{red}{\textbf{\faExclamationCircle{}} \parbox[t]{\linegoal}{#1}}}
    \newcommand{\key}          [1]{\noindent\textcolor{black}{{\textbf{{\faKey}}}} \parbox[t]{\linegoal}{#1}}
    \newcommand{\tool}         [1]{\noindent\textcolor{black}{{\textbf{{\faWrench}}}} \parbox[t]{\linegoal}{#1}}
    \newcommand{\setting}      [1]{\noindent\textcolor{black}{{\textbf{{\faCogs}}}} \parbox[t]{\linegoal}{#1}}
    \newcommand{\software}     [1]{\noindent\textcolor{black}{{\textbf{{\faFileCodeO}}}} \parbox[t]{\linegoal}{#1}}
    \newcommand{\codeblock}    [1]{\noindent\colorbox{codeGrey}{\faTerminal{} \parbox[t]{\linegoal}{\texttt{#1}}}}
    \newcommand{\option}       [1]{\noindent`\textbf{#1}'}
    \newcommand{\code}         [1]{\noindent\texttt{\sethlcolor{codeGrey}\hl{#1}}}
    \newcommand{\file}         [1]{\noindent\texttt{\faFileO{} #1}}
    \newcommand{\folder}       [1]{\noindent\texttt{\faFolderO{} #1}}