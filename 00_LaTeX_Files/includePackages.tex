\usepackage[intoc]{nomencl}
	\makenomenclature
	\renewcommand\nomgroup[1]{%
		\item[\bfseries
		\ifstrequal{#1}{A}{Constants}{%
		\ifstrequal{#1}{B}{Latin}{%
		\ifstrequal{#1}{C}{Greek}{%
		\ifstrequal{#1}{D}{Subscripts}{}}}}]%
	}
	\renewcommand{\nomname}{List of Symbols}
	\newcommand{\nomunit}[1]{\renewcommand{\nomentryend}{\hspace*{\fill}#1}}

\usepackage[acronym,nonumberlist]{glossaries}
	\glstoctrue
	\makenoidxglossaries
	\newcommand{\addacronym}[2]{\newacronym{#1}{#1}{#2}}
	\newcommand{\addterm}[2]{\newglossaryentry{#1}{name={#1},description={#2}}}
	\newcommand\generateglossary{
		\printnoidxglossary[title={\glossaryname}]}
	\newcommand\abbreviations{
		\printnoidxglossary[type=\acronymtype,title={\abbreviationsname}]}

\usepackage[utf8]{inputenc}
\usepackage[T1]{fontenc}

% Provides the comment environment (Block Comments)
%   \begin{comment}
%       <Anything goes here>
%   \end{comment}
	\usepackage{comment} 

% The basics for using math in LaTeX
	\usepackage{array}
	\usepackage{amsmath}
	\usepackage{amsfonts}
	\usepackage{amssymb}

% Used to provide underlining that can be broken across multiple lines
	\usepackage[normalem]{ulem} 

% Used for creating subfigures
	\usepackage{subcaption}

% The defacto package for including graphics
	\usepackage{graphicx}

% Used to extend or replace the table/tabular environment
	\usepackage{tabularx}  % Auto-Sizing Tables
	\usepackage{xltabular} % Multi-page Tables
	\usepackage{pdflscape} % Allows creation of landscape pages
	\usepackage{booktabs}  % Nicer Lines
	\usepackage{multicol}  % Column Merging
	\usepackage{multirow}  % Row Merging
	\usepackage{makecell}  % Allows more control over cell contents

% Used for creating consistent plots and graphs
	\usepackage{pgfplots}
		\pgfplotsset{compat=newest}
		\usepgfplotslibrary{statistics, polar}
	\usepackage{pgf-pie}  % Adds Pie charts to PGF

% Allows the inclusion PDF files
	\usepackage{pdfpages}

% Used for inserting syntax highlighted code
	\usepackage{listings}

% Used for extending the rules/control for hyphenation
% \hyp{} allows the hyphenation of manually hyphenated Words: 
%   e.g. electromagnetic\hyp{}endioscopy
	\usepackage{hyphenat}

% Used for better quoting and formatting commands
	\usepackage{csquotes}
	\usepackage{ragged2e}

% Used for including hyperlinks, document links, and better cross-references
	\usepackage{hyperref}
		\hypersetup{
			colorlinks,
			linkcolor={red!50!black},
			citecolor={blue!50!black},
			urlcolor={blue!80!black},
			pdfencoding=unicode,
			breaklinks=true
		}
		\urlstyle{same}
		\Urlmuskip=0mu plus 1mu\relax
	\usepackage[nameinlink]{cleveref}

% Used for creating the Bibliography
%   Currently set up for 
%     - IEEE style citations
%     - Compact number in-text [1 - 3] vs [1, 2, 3]
%   To change the style to APA, MLA, or other change
%     - style=ieee, and remove the citestyle=numeric-comp
	\usepackage[
		style=ieee,
		citestyle=numeric-comp,
		dashed=false,
		backend=bibtex,
		refsegment=chapter,
		sorting=none,
		defernumbers=true]{biblatex}

% EVERYTHING AFTER THIS POINT IS USED ONLY FOR FORMATTING THE EXAMPLES IN THIS
%  DOCUMENT (DO NOT USE THESE IN YOUR THESIS)
%  PLEASE DELETE THESE.
	\usepackage{float}

	\usepackage{fontawesome}
	\usepackage{lipsum}
	\usepackage{xcolor}
%% COLOUR DEFINITIONS
	\definecolor{yellow}{rgb}{0.8,0.8,0}
	\definecolor{grey}{rgb}{0.3,0.3,0.3}
	\definecolor{gold}{rgb}{0.4,0.4,0.0}
	\definecolor{cmd}{rgb}{1,0,1}
	\definecolor{cmd}{rgb}{0,0,1}
	\definecolor{env}{rgb}{0.5,0,1}
	\definecolor{pkg}{rgb}{0,0.6,0}
	\definecolor{opt}{rgb}{1,0.5,0}
	

%% NEW COMMANDS
	\newcommand{\exampleText}{\lipsum[1][1-4]}

	\newcommand{\University}{University of Alberta}
	\newcommand{\Uni}{UofA}
	\newcommand{\Faculty}{Faculty of Graduate and Post-Doctorate Studies}
	\newcommand{\Fac}{GPS}

	\newcommand{\commenting}[1]{%
		\noindent
		\begin{tabularx}{\linewidth}{@{}lX@{}}
			\textcolor{grey}{\textbf{\faCommenting}} & \textcolor{grey}{\textit{#1}}
		\end{tabularx}}

	\newcommand{\danger}[1]{%
		\noindent
		\begin{tabularx}{\linewidth}{@{}lX@{}}
			\textsc{\color{black}\colorbox{red}{\textbf{\faWarning\ DANGER}}} & {\textbf{\MakeUppercase{#1}}}
		\end{tabularx}}

	\newcommand{\warning}[1]{%
		\noindent
		\noindent
		\begin{tabularx}{\linewidth}{@{}lX@{}}
			\textsc{\color{black}\colorbox{orange}{\textbf{\faWarning\ WARNING}}} & {\textbf{#1}}
		\end{tabularx}}

	\newcommand{\caution}[1]{%
		\noindent
		\begin{tabularx}{\linewidth}{@{}lX@{}}
			\textsc{\color{black}\colorbox{yellow}{\textbf{\faWarning\ CAUTION}}} & {#1}
		\end{tabularx}}

	\newcommand{\notice}[1]{%
		\noindent
		\begin{tabularx}{\linewidth}{@{}lX@{}}
			\textsc{\color{white}\colorbox{black}{\textbf{\textit{NOTICE}}}} & {#1}
		\end{tabularx}}

	\newcommand{\alert}[1]{%
		\noindent
		\begin{tabularx}{\linewidth}{@{}lX@{}}
			\textbf{\faWarning} & {#1}
		\end{tabularx}}

	\newcommand{\note}[1]{%
		\noindent
		\begin{tabularx}{\linewidth}{@{}lX@{}}
			\textbf{\textit{Note: }} & {\textit{#1}}
		\end{tabularx}}

	\newcommand{\info}[1]{%
		\noindent
		\begin{tabularx}{\linewidth}{@{}lX@{}}
			\textcolor{blue}{\textbf{\faInfoCircle}} & \textcolor{blue}{#1}
		\end{tabularx}}

	\newcommand{\question}[1]{%
		\noindent
		\begin{tabularx}{\linewidth}{@{}lX@{}}
			\textcolor{orange}{\textbf{\faQuestionCircle}} & \textbf{#1}
		\end{tabularx}}

	\newcommand{\answer}[1]{%
		\noindent
		\begin{tabularx}{\linewidth}{@{}lX@{}}
			\-\hspace{4ex}{\textsc{\color{orange}{\textbf{\faCheckCircle}}}} & \textit{#1}
		\end{tabularx}}

	\newcommand{\qna}[2]{%
		\noindent
		\begin{tabularx}{\linewidth}{@{}lX@{}}
			\textcolor{orange}{\textbf{\faQuestionCircle}} & \textbf{#1}
		\end{tabularx}
		\noindent
		\begin{tabularx}{\linewidth}{@{}lX@{}}
			\-\hspace{4ex} & \textit{#2}
		\end{tabularx}}

	\newcommand{\tip}[1]{%
		\noindent
		\begin{tabularx}{\linewidth}{@{}lX@{}}
			\textcolor{gold}{{\textbf{{\faLightbulbO{}}}}} & \textcolor{gold}{#1}
		\end{tabularx}}
		
	\newcommand{\key}[1]{%
		\noindent
		\begin{tabularx}{\linewidth}{@{}lX@{}}
			\textcolor{black}{{\textbf{{\faKey}}}} & {#1}
		\end{tabularx}}

	\newcommand{\clouddownload}[2]{%
		\noindent
		\begin{tabularx}{\linewidth}{@{}lX@{}}
			\textbf{\faCloudDownload} & \href{#1}{#2}
		\end{tabularx}}

	\newcommand{\important}[1]{%
		\noindent
		\begin{tabularx}{\linewidth}{@{}lX@{}}
			\textcolor{red}{\textbf{\faExclamationCircle}} & \textcolor{red}{#1}
		\end{tabularx}}

	\newcommand{\tool}[1]{%
		\noindent
		\begin{tabularx}{\linewidth}{@{}lX@{}}
			\textcolor{black}{\textbf{\faWrench}} & #1
		\end{tabularx}}

	\newcommand{\setting}[1]{%
		\noindent
		\begin{tabularx}{\linewidth}{@{}lX@{}}
			\textcolor{black}{\textbf{\faCogs}} & #1
		\end{tabularx}}

	\newcommand{\software}[1]{%
		\noindent
		\begin{tabularx}{\linewidth}{@{}lX@{}}
			\textcolor{black}{\textbf{\faFileCodeO}} & #1
		\end{tabularx}}

	\newcommand{\option}[1]{%
		\noindent
		\begin{tabularx}{\linewidth}{@{}X@{}}
			`\textbf{#1}'
		\end{tabularx}}

	\newcommand{\file}[1]{%
		\noindent
		\begin{tabularx}{\linewidth}{@{}X@{}}
			\faFileO{} \texttt{#1}
		\end{tabularx}}

	\newcommand{\folder}[1]{%
		\noindent
		\begin{tabularx}{\linewidth}{@{}X@{}}
			\faFolderO{} \texttt{#1}
		\end{tabularx}}
	
	\newcommand{\str}[1]{%
		\texttt{\textcolor{pkg}{#1}}}
	
	\newcommand{\pkg}[1]{%
		\texttt{\textcolor{pkg}{#1}}}
	
	\newcommand{\cmd}[1]{%
		\texttt{\textcolor{cmd}{\textbackslash#1}}}
	
	\newcommand{\opt}[1]{%
		\texttt{\textcolor{opt}{#1}}}
	\newcommand{\oopt}[1]{%
		\texttt{[\textcolor{opt}{#1}]}}
	\newcommand{\mopt}[1]{%
		\texttt{\{\textcolor{opt}{#1}\}}}
	
	\newcommand{\env}[1]{%
		\texttt{\textcolor{env}{#1}}}
		
	\newcommand{\benv}[1]{%
		\texttt{\cmd{begin}\{\textcolor{env}{#1}\}}}
	\newcommand{\eenv}[1]{%
		\texttt{\cmd{end}\{\textcolor{env}{#1}\}}}
	