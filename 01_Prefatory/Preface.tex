\preface{
    \textbf{If you need assistance on writing the preface, ask your supervisor.
    Your supervisor must review and verify the preface before it becomes part of the final version of the thesis.}

    A preface is a mandatory component of a thesis, regardless of thesis format, \underline{when} a thesis contains journal articles authored or co-authored by the student (including an accepted paper that is forthcoming at the time of thesis submission). 
    A preface is \underline{also} a mandatory component when the research conducted for the thesis required ethics approval. 
    A preface remains optional if there is no inclusion of journal articles and/or no need for ethics approval.
    
    When required because a thesis contains journal articles, the preface serves as a place for the student to include a statement indicating his or her contribution to the journal articles, such as the identification and design of the research program, the performance of the various parts of the research (including the collection of data, construction of any necessary apparatus, and the performance of experiments), and the analysis of the research data. 
    If any of the work presented in the thesis has led to any publications (accepted or published), these publications must be listed clearly in the preface with their bibliographical details and an indication as to where in the thesis this work is located (e.g. state in which chapter or chapters). 
    For jointly authored publications, indication must also be given as to the relative contributions of the collaborators and co-authors, and a statement as to the proportion of research and writing conducted by the student. 
    Note that permission may be needed if the co-authors hold the copyright in these publications. 
    If ethics approval was required for the research, a statement to this effect must be included in the preface with the details of the approval that was granted. 
    
    Note that the inclusion of a preface does not excuse a student from failing to acknowledge the contributions of others in the body of the thesis, as per the University's Research and Scholarship Integrity Policy and the Code of Student Behaviour. One would still expect to see footnotes, endnotes or in-text references within the thesis acknowledging the works. Acknowledgements, such as thanks to the supervisor and supervisory committee members, to colleagues, lab mates and friends, and to family, do not appear in the preface. 
    
    \textbf{Examples of several prefaces are given in Appendix B and are also available from the FGSR website.}
}
   
  % Optional
    %\preface{%
      %This thesis is an original work by `Your name here'. No part of this thesis has been previously published.
    %}
    
  % Mandatory due to research ethics approval
    %\preface{
      %This thesis is an original work by `Your name here'. 
      %The research project, of which this thesis is a part, received research ethics approval from the University of Alberta Research Ethics Board, Project Name “TITLE”, No. 12345, DATE.
    %}
    
  % Mandatory due to collaborative work
    %\preface{%
      %Some of the research conducted for this thesis forms part of an international research collaboration, led by Professor R.C. Smith at the University of Hogwarts, with Professor D.R. Brown being the lead collaborator at the University of Alberta. 
      %The technical apparatus referred to in chapter 3 was designed by myself, with the assistance of M.C. White and Professor D.R. Brown. 
      %The data analysis in chapter 4 and concluding analysis in chapter 5 are my original work, as well as the literature review in chapter 2.
      %
      %Chapter 3 of this thesis has been published as J.D. Doe, M.C. Smith, and R.C. Smith, “Theoretical Responses to Rays in the Gamma System,” Journal of Scientific Affairs, vol. 165, issue 3, 459-475. 
      %I was responsible for the data collection and analysis as well as the manuscript composition. 
      %M.C. Smith assisted with the data collection and contributed to manuscript edits. R.C. Smith was the supervisory author and was involved with concept formation and manuscript composition. 
    %}