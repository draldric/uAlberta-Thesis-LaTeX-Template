\chapter{Document Structure and Formatting}\label{ch:documentstructure}
	A thesis for the \University\ can consist of many different parts that come together to create the final document.
	These will include the Title Page, Abstract, and other Prefatory pages; and the chapters, sections, and paragraphs; as well as the content, figures, tables, and other content you would like to include.
	In the rest of this \nameCref{ch:documentstructure}, we will look more specifically on the inclusion of the prefatory pages and how to structure your document with the various chapter, section, and paragraph commands.
	
	\section{Title Page, Abstract, and Other Prefatory Pages}
		To create a title page in \LaTeX, you can use the \cmd{maketitle} command after providing the necessary title, author, and date information. 
		This is usually performed in the following way:

		\begin{lstlisting}[style=LaTeXStyle]
\title{Your Thesis Title}
\author{Your Name}
\date{\today}

\begin{document}
	\maketitle
\end{document}
		\end{lstlisting}

		For a thesis at the \University, there is more information that is required, as well as a few more pages that are required (Abstract and Preface) and some that are optional (Quote, Dedication, and Acknowledgements), but all of them have specific formatting requirements.
		\Fac\ even provides a few different documents that attempt to provide the necessary information to create these pages\cite{FGPS2024}.
		To aid you in the creation of these pages I have created a few new pairs of macros (one sets the text for the section, the other one generates the page with the text):
		\note{I chose to separate the two macros to better compartmentalize the thesis (each prefatory page has its own file), as well as allow an easy way to include and exclude each of the prefatory pages via commenting}
		\begin{itemize}
			\item Abstract
				\begin{itemize}
					\item \cmd{abstracttext}\mopt{Abstract Text Goes Here!}
					\item \cmd{makeabstract}
				\end{itemize}
			\item Preface
				\begin{itemize}
					\item \cmd{preface}\mopt{Preface Text goes here.}
					\item \cmd{makepreface}
				\end{itemize}
			\item Quote
				\begin{itemize}
					\item \cmd{thesisquote}\mopt{Quote Text goes here.}|
					\item \cmd{makequote}
				\end{itemize}
			\item Dedication
				\begin{itemize}
					\item \cmd{dedication}\mopt{Dedication Text goes here}|
					\item \cmd{makededication}
				\end{itemize}
			\item Acknowledgement
				\begin{itemize}
					\item \cmd{acknowledgementtext}\mopt{Acknowledgement Text goes here.}|
					\item \cmd{acknowledgements}
				\end{itemize}
		\end{itemize}

		\subsection{Title Page}
			The thesis Title Page has a few more fields to be filled in than a regular \LaTeX\ document.
			These include \cmd{degree}, \cmd{specialization}, \cmd{deptfac}, and \cmd{convocationdate}.
			An example of how to fill these in can be seen in the original \LaTeX\ code (\path{ualberta.tex}) or in \Cref{lst:TitlePage}.

			Most of the fields are fairly self explanatory, however, to be extra clear as to what needs to be included I will now provide an explanation of each field:

			\begin{table}[H]
				\centering
				\caption{Title Page Macro Definitions and Examples}
				\label{tab:titlePage}
				\begin{tabularx}{0.9\linewidth}{cCC}
					\toprule
						\textbf{Field} & \textbf{Description} & \textbf{Example}\\
					\midrule
						\cmd{title}  & The Title of your Thesis. & The Perfect Thesis Title That is Perfectly Captivating\\
						\cmd{author} & Your Full Name. & Daniel Ryan Aldrich\\
						\cmd{degree} & Degree or one of the premade macros (note they are \textbf{fairly} case insensitive) \textit{e.g.}, \cmd{MSc}. & Master of Science or \cmd{Msc}\\
						\cmd{specialization} & Specialization, otherwise, leave it blank. & Applied Math\\
						\cmd{department} & Department, or if you are non-departmentalized, leave this blank. & Mechanical Engineering\\
						\cmd{faculty} & If you are non-departmentalized, fill this in, otherwise, leave this blank. & \\
						\cmd{convocationdate} & The year in which you will \textbf{convocate}. & 2024\\%TODO: CONFIRM THIS
					\bottomrule
				\end{tabularx}
			\end{table}

			\begin{lstlisting}[float=ht,caption=Example of How to Set Title Page Info,label=lst:TitlePage,style=LaTeXStyle,basicstyle=\scriptsize\ttfamily,]
%%%%%%%%%%%%%%%%%%%%%%%%%%%%%%%%%%%%%%%%%%%%%%%%%%%%%%%%%%%%%%%%%%%%%%%%%%%%%%%%
%                  TITLE PAGE AND FRONTMATTER INFORMATION                      %
%%%%%%%%%%%%%%%%%%%%%%%%%%%%%%%%%%%%%%%%%%%%%%%%%%%%%%%%%%%%%%%%%%%%%%%%%%%%%%%%
% TITLE PAGE INFO
  \title{Thesis Title}                         % Title of your Thesis
  \author{First Middle Last}                   % Your Full Name
  \degree{\insertlatexfile{selectDegree}}      % Uncomment Degree in file
  \specialization{}                            % Leave blank if none
  \deptfac{\insertlatexfile{selectDepartment}} % Uncomment Department in file
  \convocationdate{\the\year}                  % Convocation Year
			\end{lstlisting}
			
			\notice{This class, though it is designed to be used with at \University, it is not exclusive to the \University.
			To customize this to your specific institution, use the command \cmd{university}\mopt{YOUR INSTITUTION NAME} before the \cmd{maketitle} command.
			Note: this template still might not match your institution's guidelines on formatting a thesis.}

		\subsection{Abstract}\label{abstract}
			This is probably one of the simplest sections that need to be created for you thesis.
			Mainly because the content that can be included is so restrictive.
			Further, the limit on the word count makes this section quite short.

			\warning{Just because of the above does not mean this section won't be difficult to write. It will be. But the best piece of advice I can offer is write this section last. In fact, just to prove a point, I will add an appendix to this document that will go over the actual writing of this section.}

		\subsection{Preface}\label{preface}
			The following Sections will provide you with examples of how to include specific elements into your preface.
			The examples in the following sections have been modified from \fullcite{FGPS2024}.
			Due to the ``Appendix for Thesis Formatting Guidelines'' being generated using ChatGPT3.5 there are some inconsistencies within the document and the examples provided; while these are fine for providing a general idea of what to include, this goes against the ideology of this document.
			To combat this I have taken the liberty of adapting the examples to better blend into the style of this document and to increase the consistency, replacing some the generated examples with real examples where possible.

			Full credit still goes to the \University's Faculty of Graduate and Postdoctoral Studies for openly providing the document and I would strongly suggest anyone writing their thesis at the \University\ to read the updated documentation that they have put out this year (2024).
			This new documentation provides much more guidance than the previous iteration: \fullcite{FGSR2016}.

			Before diving right into the examples in \Cref{preface:REA,preface:CW,preface:PPW,preface:AI,preface:CF,preface:DNA}, we will explore some of the still applicable guidelines for including your preface. First of which is a small change: since the FGSR became the \Fac, the requirement of a preface has become \textbf{mandatory} regardless of the contents of the thesis.

			When a thesis contains journal articles, the preface serves as a place for the student to include a statement indicating his or her contribution to the journal articles. 
			If any of the work presented in the thesis has led to any publications (accepted or published), these publications must be listed clearly in the preface with their bibliographical details and an indication as to where in the thesis this work is located (\textit{e.g.} state in which Chapter or Chapters). 
			For jointly authored publications\footnote{Permission may be needed if the co-authors hold the copyright in these publications.}, indication must also be given as to the relative contributions of the collaborators and co-authors, and a statement as to the proportion of research and writing conducted by the student.
			If ethics approval was required for the research, a statement to this effect must be included in the preface with the details of the approval that was granted. 

			\warning{The inclusion of a preface does not excuse a student from failing to acknowledge the contributions of others in the body of the thesis, as per the University's Research and Scholarship Integrity Policy and the Code of Student Behaviour. One would still expect to see footnotes, endnotes, or in-text references within the thesis acknowledging the works.}

			\caution{Acknowledgements, such as thanks to the supervisor and supervisory committee members, to colleagues, lab mates and friends, and to family, \textbf{do not} appear in the preface.}

			\notice{If you need assistance on writing the preface, ask your supervisor.Your supervisor must review and verify the preface before it becomes part of the final version of the thesis.}

			\note{Before preceding I would like to emphasize that the following examples are are not a one-and-done, you might need to combine elements from multiple of the following sections to build an appropriate preface for your thesis.}

			\info{If \textbf{none} of the following apply \textbf{make sure} to include a preface in the style of the one shown in \Cref{preface:DNA}.}

			\tip{Fun fact before you read on, Professor C. Ayranci who was mentioned in a few of the examples provided by ``Appendix for Thesis Formatting Guidelines'' was in-fact the supervisor for my Master's thesis.}

			\subsubsection{Research Ethics Approval}\label{preface:REA}
				If your thesis required you to get Research Ethics Approval, then you should include a preface based on the one here.

				\begin{quote}
					\enquote{%
						This thesis is an original work by Daniel R.\ Aldrich. 
						The research project, of which this thesis is a part, received research ethics approval from the \University\ Research Ethics Board 3, Project Name "Etymologies and Entomologies: Unraveling the Threads of Language and Ecology", No. 12345, February 23, 1993.}%TODO Change This to a fun project Name.
				\end{quote}

			\subsubsection{Collaborative Work}\label{preface:CW}
				If your thesis required you to work collaboratively with other, organizations, researchers or otherwise, then you should include a preface based on the one here.

				\begin{quote}
					\enquote{%
						Some of the research conducted for this thesis forms part of an international research collaboration, led by Professor T. Raivio at the University of Hogwarts, with Professor S. Agrawal being the lead collaborator at the \University. 
						The technical apparatus referred to in \Cref{ch:documentstructure} was designed by myself, with the assistance of Professor A. Shiri and Professor C. Ayranci. 
						The data analysis in \Cref{ch:figureandtables} and concluding analysis in \Cref{ch:plotsandgraphs} are my own work, as well as the literature review in \Cref{ch:gettingstarted}.}
				\end{quote}

			\subsubsection{Previously Published Material}\label{preface:PPW}
				If your thesis is based on or includes work previously published by yourself, or work you co-authored, then you should include a preface based on the one here.

				\begin{quote}
					\enquote{%
						\Cref{ch:gettingstarted} of this thesis has been published as \fullcite{Aldrich2017}. 
						I was responsible for the programming, data collection, and analysis as well as the manuscript composition. C. Ayranci assisted with the data collection by providing access to the Multipurpose Composites Group's Lab and Equipment, and contributed to manuscript edits. D. Nobes was the supervisory and corresponding author and was involved with concept formation and manuscript composition.}
				\end{quote}

			\subsubsection{Use of Artificial Intelligence (AI)}\label{preface:AI}
				If your thesis used AI to help you outline, or otherwise write sections of including help with analyzing, summarizing, \textit{etc.}, then you should include a preface based on the one here.

				\begin{quote}
					\enquote{%
						The generative artificial intelligence application or Large Language Model ChatGPT 3.5 was used for data analysis, summarization, synthesis, and simulation in \Cref{ch:documentstructure} of this thesis, as well as to generate a preliminary draft of the literature review in \Cref{ch:Introduction}.}
				\end{quote}

			\subsubsection{Receiving of Competitive Funding}\label{preface:CF}
				If you received competitive funding used for your thesis, education, or research then you should include a preface based on the one here.

				\begin{quote}
					\enquote{%
						This work was supported by a Doctoral Fellowship from the Social Sciences and Humanities Research Council, a grant from the Entomological Association of Edmonton, and the National Scholarship Council of Narnia.}
				\end{quote}
				
			\subsubsection{Previous Prefaces Did Not Apply}\label{preface:DNA}
				If your thesis did not fall into any of the categories shown previously, then you should include a preface based on the one here.
				
				\note{This was previously an \textbf{Optional} inclusion in your thesis, however, since the FGSR transitioned to the \Fac and have released new guidelines for formatting your thesis, this has now become \textbf{mandatory}.}

				\begin{quote}
					\enquote{%
						This thesis is an original work by \texttt{YOUR FULL NAME}. 
						No part of this thesis has been previously published.}
				\end{quote}

		\subsection{Dedication or Quotations}\label{quote}\label{dedication}
			Dedications or Quotations are limited to maximum of one page.
			
			In the template provided you may fill in the appropriate Quote and Dedication fields in the document \path{01_Prefatory/Quotes_Dedications.tex}.
			These can be filled out if you plan to include them or not.
			To control whether they show up in the document you will want to un-comment \textbf{ONLY ONE} of the following commands in the \path{ualberta.tex} file.

			\begin{lstlisting}[float=ht,caption=Quote and Dedication Inclusion Options (un-comment only one),label=lst:QuotesAndDedication,style=LaTeXStyle,basicstyle=\scriptsize\ttfamily,]
  %\makequote                  % Creates the Quote Page
  %\makededication             % Creates the Dedication Page
  %\makededicationandquote     % Creates the Quote/Dedication Page
			\end{lstlisting}

		\subsection{Acknowledgements}\label{acknowledgement}
			The Acknowledgements is a strongly recommended, but not a strictly mandatory section of the thesis.
			For a thesis at the \University, an Acknowledgement page must be no more than two (2) pages in length, and \textbf{must} be double-spaced.
	
			The Acknowledgements page serves as a place within a thesis where students may wish to acknowledge the provision of funding from third parties, such as an external scholarship bodies, research granting agencies, and foreign governments. 
			It is also appropriate to recognize the assistance provided by the supervisor and members of the supervisory committee.
			Additional, you can recognize any help or support that you have received from family, friends, classmates, lab-mates, or any other individual that was able to help you in any way.
			A clear example of this would be the inclusion of the following, assuming that his template, class, or guide helped you in any significant way\footnote{Please be aware that it is not necessary to include this statement in any way, shape, or form it is only provided as an example on how to include these type of statement; this template/guide is provided in good faith with the \textbf{only} expectation being that you \textbf{will} share it with your fellow colleagues, supervisor, and or anyone else that may benefit from this.}:
			
			\begin{quote}
				\enquote{I would like to thank Daniel R. Aldrich for his continuing contributions to the \University, and for his work within the Graduate Student Community. More specifically, I would like to acknowledge the work that he has put into creating the \LaTeX\ template that this thesis was created in and the ongoing support that he provides to the students at the \University\ in formatting their theses.}
			\end{quote}

		\subsection{Table of Contents}\label{toc}
			Include chapter headings and 2--4 levels of subheadings. 
			This template will automatically include all required chapters including the correct ones from the prefatory pages.
			However, due to stylistic considerations you may want more or less levels included the in the Table of Contents.
			To accomplish this I have created the command near the top of the \path{ualberta.tex} file that allows you to set both the depth of the Numbered Headings and the level of depth for the Table of Contents.
			To do this change the number shown in \Cref{lst:tocLevels} to 2, 3, or 4.
			\begin{lstlisting}[float=ht,caption=Set Numbered Heading and ToC Level,label=lst:tocLevels,style=LaTeXStyle,basicstyle=\scriptsize\ttfamily,]
% Option to change the Level of subheading included in the Table of Contents
%  This should be set at 2, 3, or 4 (As per \Fac)
  \settoclevel{3}
			\end{lstlisting}

			\tip{For the \cmd{settoclevel}\mopt{n} command, setting \opt{n} to \opt{2} will included everything down to \texttt{subsection}, \opt{3} will include everything down to \texttt{subsubsection}, and \opt{4} will include everything down to \texttt{paragraph} in the Table of Contents.}

		\subsection{Lists of Figures, Tables, ...}\label{listsof}
			Include a separate list, beginning on a new page, for each kind of non-textual item appearing in the body of the thesis (one list for tables, another for illustrations, etc.).
			Lists can be in any order.
			These are included with the following set of commands:
			\begin{itemize}
				\item \cmd{listoffigures}
				\item \cmd{listoftables}
				\item \cmd{listofplates}
			\end{itemize}
			
			\note{None of these commands need to be commented out, they can stay uncommented, and as long as \LaTeX\ is run at minimum twice (as is required for BiBTeX, and any reference updates) they will update themselves to be included or excluded automatically.}

	\section{Nomenclature, Glossary, \& Acronyms}\label{nomenclature}\label{glossary}\label{acronyms}
		\subsection{Lists of Symbols/Abbreviations}
			

		\subsection{Glossary of Terms}
			

	\section{Chapters, Sections, Subsections, \textit{etc.}}
		Organize your document hierarchically using chapters, sections, subsections, \textit{etc}.
		These structures all utilize the base macros from \LaTeX{} including:
			\begin{itemize}
				\item \cmd{chapter}\mopt{Chapter Heading}|, 
				\item \cmd{section}\mopt{Section Heading}|, 
				\item \cmd{subsection}\mopt{Subsection Heading}|, 
				\item \cmd{subsubsection}\mopt{Sub-Subsection Heading}|, 
				\item \cmd{paragraph}\mopt{Paragraph Heading}|, 
				\item \cmd{subparagraph}\mopt{Subparagraph Heading}|. 
			\end{itemize}
		\tip{For writing your thesis, it is strongly recommended that one outlines the thesis using these commands first, while also added in a small description of what that chapter, section, \textit{etc}., should accomplish. This will help you stay organized and on track. Remembering that you can use comments, \%, to hide these descriptions when you start to fill in your content.}

	\section{Page Layout and Margins}
		\warning{While one can adjust the values using the commands provided by the following packages, unless you really know \LaTeX\ inside and out this should be avoided.
			Everything provided in these files are aimed at making writing your thesis as easy as possible.}

		You can customize the layout and margins of your document using the \pkg{geometry} package. 
		Additionally, you can use the \pkg{titlesec} package to customize the formatting of chapter and section titles.