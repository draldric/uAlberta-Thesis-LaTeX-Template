\chapter{Document Structure and Formatting}
	A thesis for the University of Alberta can consist of many different parts that come together to create the final document.
	These will include the Title Page, Abstract, and other prefatory pages; and the chapters, sections, and subsections.
	In the following sections, we will look into how we can add these different sections and how to manipulate them too.
	
	\section{Title Page, Abstract, and Other Prefatory Pages}
		To create a title page in \LaTeX{}, you can use the \lstinline|\maketitle| command after providing the necessary title, author, and date information. 
		This is usually performed in the following way:

		\begin{lstlisting}[style=LaTeXStyle]
		\title{Your Thesis Title}
		\author{Your Name}
		\date{\today}

		\begin{document}
			\maketitle
		\end{document}
		\end{lstlisting}

		For a thesis at the University of Alberta, there are a few more pages that are required (Abstract and Preface) and some that are optional (Quote, Dedication, and Acknowledgements), but all of them have specific formatting requirements.
		To aid you in the creation of these pages a few new macros have been provided.
		\begin{itemize}
			\item \lstinline|\abstracttext{Abstract Text goes here.}|
			\item \lstinline|\preface{Preface Text goes here.}|
			\item \lstinline|\thesisquote{Quote Text goes here.}|
			\item \lstinline|\dedication{Dedication Text goes here}|
			\item \lstinline|\acknowledgementtext{Acknowledgement Text goes here.}|
		\end{itemize}

		\subsection{Title Page}
			The thesis Title Page has a few more fields to be filled in than a regular \LaTeX{} document.
			These include \lstinline|\degree|, \lstinline|\specialization|, \lstinline|\department|, \lstinline|\faculty|, and \lstinline|\convocationdate|.
			An example of how to fill these in can be seen in the original \LaTeX{} code (\path{ualberta.tex}) or in \Cref{lst:TitlePage}.

			Most of the fields are fairly self explanatory, however, to be extra clear as to what needs to be included I will now provide an explanation of each field.:

			\begin{table}[H]
			\caption{Title Page Macro Definitions and Examples}
			\label{tab:titlePage}
			\begin{tabularx}{0.9\textwidth}{cCC}
				\toprule
					\textbf{Field} & \textbf{Description} & \textbf{Example}\\
				\midrule
					\lstinline|\\title|  & The Title of your Thesis. & The Perfect Thesis Title That is Perfectly Captivating\\
					\lstinline|\\author| & Your Full Name. & Daniel Ryan Aldrich\\
					\lstinline|\\degree| & Degree or one of the premade macros (note they are \textbf{not} case sensitive) \textit{e.g.}, \lstinline|\\MSc|. & Master of Science or \lstinline|\\Msc|\\
					\lstinline|\\specialization| & Specialization, otherwise, leave it blank. & Applied Math\\
					\lstinline|\\department| & Department, or if you are non-departmentalized, leave this blank. & Mechanical Engineering\\
					\lstinline|\\faculty| & If you are non-departmentalized, fill this in, otherwise, leave this blank. & \\
					\lstinline|\\convocationdate| & The year in which you will \textbf{convocate}. & 2024\\
				\bottomrule
			\end{tabularx}
			\end{table}

			\begin{lstlisting}[float=h,caption=Example of How to Set Title Page Info,label=lst:TitlePage,style=LaTeXStyle,basicstyle=\scriptsize\ttfamily,]
			%%%%%%%%%%%%%%%%%%%%%%%%%%%%%%%%%%%%%%%%%%%%%%%%%%%%%%%%%%%%%%%%%%%%%%%%%%%%%%%%
			%                  TITLE PAGE AND FRONTMATTER INFORMATION                      %
			%%%%%%%%%%%%%%%%%%%%%%%%%%%%%%%%%%%%%%%%%%%%%%%%%%%%%%%%%%%%%%%%%%%%%%%%%%%%%%%%
			% TITLE PAGE INFO
			  \title{Thesis Title}              % Title of your Thesis
			  \author{First Middle Last}        % Your Full Name
			  \degree{\MSc}                     % \MSc, \PhD, \MA, \MEd, \MBA, \MAc, \MFM, \MN, \LLM, or \MMus
			  \specialization{}                 % Leave blank if none
			  \department{Example Department}   % Fill in the Department unless you are non-Departmental
			  \faculty{}                        % Leave blank unless non-Departmental
			  \convocationdate{2023}            % Convocation Year
			\end{lstlisting}

	\section{Chapters, Sections, and Subsections}
		Organize your document hierarchically using chapters, sections, subsections, \textit{ect}.
		These structures all utilize the base macros from \LaTeX{} including:
			\begin{itemize}
				\item \lstinline|\chapter{Chapter Heading}|, 
				\item \lstinline|\section{Section Heading}|, 
				\item \lstinline|\subsection{Subsection Heading}|, 
				\item \lstinline|\subsubsection{Sub-Subsection Heading}|, 
				\item \lstinline|\paragraph{Paragraph Heading}|, 
				\item \lstinline|\subparagraph{Subparagraph Heading}|. 
			\end{itemize}
		\tip{For writing your thesis, it is strongly recommended that one outlines the thesis using these commands first, while also added in a small description of what that chapter, section, \textit{ect.}, should accomplish. This will help you stay organized and on track. Remembering that you can use comments, \%, to hide these descriptions when you start to fill in your content.}

	\section{Page Layout and Margins}
		\warning{While one can adjust the values using the commands provided by the following packages, unless you really know \LaTeX{} inside and out this should be avoided.
		Everything provided in these files are aimed at making writing your thesis as easy as possible.}\\

		You can customize the layout and margins of your document using the \texttt{geometry} package. 
		Additionally, you can use the \texttt{titlesec} package to customize the formatting of chapter and section titles.


