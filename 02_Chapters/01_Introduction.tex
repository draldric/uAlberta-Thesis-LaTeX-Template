\chapter{Introduction}\label{ch:Introduction}\section{Background}
	As a graduate student from the University of Alberta, I understand the daunting task that is associated with writing a Thesis that conforms to the guidelines outlined in the FGSR Minimum Thesis Formatting Requirements. 
	It can also be very frustrating to write long, equation and figure heavy, document in a word processor that is prone to crashes, file corruption, seemingly random changes to the formatting, and that do not output a document in the required PDF/A format for submission to FGSR.
	
	Due to these problems, lots of students attempt to use an alternative to traditional word processors: \LaTeX. 
	
	\LaTeX{} allows students/researchers to focus on either the writing of the document or the formatting. 
	Because the writing is separated from the formatting, the writing of the documents can be performed in much more lightweight text editors, or \TeX{} editors (that also allow for the compilation of the documents). 
	These editors are usually able to save the work after every keystroke and due to the plaintext nature, are not generally susceptible to file corruption.
	\LaTeX{} has the added benefit of providing a consistent and professional look and feel throughout the document. 

	\section{Objectives}
		The main objectives of this thesis are:
		\begin{enumerate}
			\item To provide a comprehensive guide on writing a thesis using \LaTeX{}.
			\item To assist students and researchers in mastering the nuances of \LaTeX{} document preparation.
			\item To showcase best practices for structuring and formatting a thesis in \LaTeX{}.
		\end{enumerate}

	\section{Scope and Limitations}
		While there are existing templates for writing a thesis for the University of Alberta in \LaTeX{}, there does not appear to be a template for \LaTeX{} that provides students all the information required to write an outstanding thesis.
		This template/document aims to defeat this shortcoming by providing all the necessary information to create a well structured thesis, as well as providing examples to assist in the formatting of documents written in \LaTeX{}.
		This thesis focuses on the following aspects:
		\begin{itemize}
			\item Installation and basic usage of \LaTeX{}.
			\item Document structure and formatting.
			\item Inclusion of figures and tables.
			\item Handling mathematical equations.
			\item Citations and references using BibTeX.
			\item Introduction to advanced topics and recommended packages.
		\end{itemize}

		However, it does not cover advanced \LaTeX{} programming or extensive customization of document classes.
		Mainly because I did the heavy lifting for you; the class file, \path{ualberta.cls}, provides all the major document requirements while this document provides the references of how to include all the bits and bobs that one might what in a thesis.

	\section{Organization of the Thesis}
		The thesis is organized into several chapters, each addressing a specific aspect of writing a thesis in \LaTeX{}. 
		The breakdown is as follows:
		\begin{itemize}
			\item \textbf{Chapter 2:} Getting Started with \LaTeX{}
			\item \textbf{Chapter 3:} Document Structure and Formatting
			\item \textbf{Chapter 4:} Figures and Tables
			\item \textbf{Chapter 5:} Mathematical Equations
			\item \textbf{Chapter 6:} Citations and References
			\item \textbf{Chapter 7:} Advanced Topics and Recommended Packages
			\item \textbf{Chapter 8:} Conclusion
		\end{itemize}

		Each chapter provides detailed information, examples, and recommendations to help the reader navigate the \LaTeX{} document preparation process effectively.

	\section{Summary}
		This chapter introduced the background, objectives, scope, and organization of the thesis. 
		The subsequent chapters delve into specific topics, providing practical guidance and examples for mastering the art of writing a thesis in \LaTeX.
\begin{comment}
  The first page of the introduction is marked as page ``1'' and then the pages that follow are numbered sequentially.

  The minimum academic requirements for the text of a thesis are an introduction, followed by the presentation of the research in a manner suitable for the field, and a conclusion.

  The introduction must outline the thesis, problem, hypothesis, questions or goals of the research.
  It must provide a clear statement of the research question(s).
  The conclusion must highlight the student’s contribution to knowledge, providing conclusions with respect to the problem, hypothesis or goals of the research.
  In all theses, regardless of format, the body or chapters of the thesis contain methodology, research results, and scholarly discussion in accordance with the norms of the academic discipline.

  The University of Alberta encourages students to publish.
  Thus, one or more chapters of a thesis may contain published material if permitted by the regulations of your department (or Faculty if non-departmentalized) governing your specific degree program.
  It is a matter for individual graduate programs to develop specific guidance for students and supervisors, as well as supervisory committee members, with such specific guidance likely to reflect the needs of the particular field or academic discipline.

  For example, individual graduate programs that continue to prefer the traditional monograph-style thesis may have department-specific rules on such matters as maximum length (taking into account the burden on an external examiner to read a thesis that is over 100,000 words).
  Individual graduate programs that opt for a multiple-manuscript or journal-article format, or choose to accept both traditional and paper-based formats, or a combination of both, must have department-specific guidelines in place to address potential student and supervisor queries.
  For example, the department-specific guidance needs to address what counts as a publication (\textit{e.g. } self-publication, any journal, or only journals listed in a particular source); whether a paper-based thesis can include both published and accepted papers; whether the student must be the first author for a multiple-authored publication to be acceptable for inclusion within the thesis; and whether some form of connecting text is needed to link the papers beyond the introduction (and if so, what form or forms of connectors are acceptable).

  \section{Motivation}\label{sec:Motivation}
    \lipsum[1-3]
  \section{Thesis Objectives}\label{sec:thesisObjective}
    \lipsum[14-16]
  \section{Thesis Outline}\label{sec:thesisOutline}
    \lipsum[17]
\end{comment}