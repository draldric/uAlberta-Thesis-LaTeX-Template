\chapter{Introduction}\label{ch:Introduction}
  The first page of the introduction is marked as page ``1'' and then the pages that follow are numbered sequentially.

  The minimum academic requirements for the text of a thesis are an introduction, followed by the presentation of the research in a manner suitable for the field, and a conclusion.

  The introduction must outline the thesis, problem, hypothesis, questions or goals of the research.
  It must provide a clear statement of the research question(s).
  The conclusion must highlight the student’s contribution to knowledge, providing conclusions with respect to the problem, hypothesis or goals of the research.
  In all theses, regardless of format, the body or chapters of the thesis contain methodology, research results, and scholarly discussion in accordance with the norms of the academic discipline.

  The University of Alberta encourages students to publish.
  Thus, one or more chapters of a thesis may contain published material if permitted by the regulations of your department (or Faculty if non-departmentalized) governing your specific degree program.
  It is a matter for individual graduate programs to develop specific guidance for students and supervisors, as well as supervisory committee members, with such specific guidance likely to reflect the needs of the particular field or academic discipline.

  For example, individual graduate programs that continue to prefer the traditional monograph-style thesis may have department-specific rules on such matters as maximum length (taking into account the burden on an external examiner to read a thesis that is over 100,000 words).
  Individual graduate programs that opt for a multiple-manuscript or journal-article format, or choose to accept both traditional and paper-based formats, or a combination of both, must have department-specific guidelines in place to address potential student and supervisor queries.
  For example, the department-specific guidance needs to address what counts as a publication (\textit{e.g. } self-publication, any journal, or only journals listed in a particular source); whether a paper-based thesis can include both published and accepted papers; whether the student must be the first author for a multiple-authored publication to be acceptable for inclusion within the thesis; and whether some form of connecting text is needed to link the papers beyond the introduction (and if so, what form or forms of connectors are acceptable).

  \section{Motivation}\label{sec:Motivation}
    \lipsum[1-3]
  \section{Thesis Objectives}\label{sec:thesisObjective}
    \lipsum[14-16]
  \section{Thesis Outline}\label{sec:thesisOutline}
    \lipsum[17]
