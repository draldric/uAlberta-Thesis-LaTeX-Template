\chapter{Introduction}\label{ch:Introduction}
	\section{Background}
		As a graduate student from the \University, I am familiar with the challenging task of writing a thesis that adheres to the \Fac\ Minimum Thesis Formatting Requirements. 
		Using a traditional word processor to create a long document filled with equations and figures can be frustrating due to frequent crashes, file corruption, unpredictable formatting changes, and the inability to output a document in the required PDF/A format for submission to \Fac.
		
		To overcome these issues, many students turn to \LaTeX\ as an alternative to conventional word processors. 
		
		\LaTeX\ allows students and researchers to focus separately on the content and the formatting of their documents. 
		Because the writing is independent of the formatting, documents can be written in lightweight text editors or \LaTeX\ editors, which also facilitate the compilation of the documents. 
		These editors can often save work after every keystroke, and due to the plaintext format, they are less prone to file corruption.
		Moreover, \LaTeX\ ensures a consistent and professional appearance throughout the document. 

	\section{Objectives}
		The main objectives of this thesis are:
		\begin{enumerate}
			\item To provide a comprehensive guide on writing a thesis using \LaTeX.
			\item To assist students and researchers in mastering the nuances of \LaTeX\ document preparation.
			\item To showcase best practices for structuring and formatting a thesis in \LaTeX.
		\end{enumerate}

	\section{Scope and Limitations}
		Although there are existing templates for writing a thesis in \LaTeX\ for the \University\ (see \Cref{tab:OtherTemplates} for a list of available templates), none seem to provide all the necessary information for creating an outstanding thesis.
		Most templates apply ``band-aid'' solutions to existing classes, such as \opt{report} or \opt{book}, offering a customized title page and methods for including prefatory pages.
		However, these templates often fall short by not providing tips and best practices on how to include the various sections and parts that make up a thesis.
		They also fail to offer a solid foundation for those who are new to \LaTeX.
		Many of these templates involve extensive patching and fixing, resulting in a large \textit{preamble} section at the beginning of the template that can be confusing to new \LaTeX\ users and add to the already steep learning curve.
		
		\begin{landscape}
			\begin{table}[p]
				\centering
				\small
				\begin{tabularx}{\linewidth}{ccL}
					\toprule
						Developer & Last Updated & Link to Template Source\\
					\midrule
						Shivam Garg & May 29, 2023 & \url{https://github.com/svmgrg/ualberta_thesis_template}\\
						Henry Brausen & Feb 11, 2022 & \url{https://github.com/henrybrausen/thesis_template}\\
						Bernard Llanos & Oct 05, 2019 & \url{https://drive.google.com/file/d/1wKS8fu5e6qiVDRt0VUzEtlW8p7uMyz1T/view?usp=sharing}\\
						John Bowman & Sep 30, 2019 & \url{https://github.com/vectorgraphics/uofathesis}\\
						\makecell{Hongtao Yang\\\&\\Benjamin Bernard} & Sep 28, 2017 & \url{https://github.com/adrs0049/ThesisTemplate}\\
						\makecell{GAME\\\&\\Hongtao Yang} & Feb 03, 2016 & \url{https://www.ualberta.ca/computing-science/media-library/grad/candidacy-template-tex.tex}\\
						Steven Taschuk & Mar 21, 2012 & \url{https://github.com/stebulus/ualberta-math-stat-templates/tree/master/thesis}\\
						CMENG & Jul 19, 1999 & \url{https://sites.ualberta.ca/CMENG/research/new-control/stythes.html}\\
					\bottomrule
				\end{tabularx}
				\caption{List of Other Available Templates.}
				\label{tab:OtherTemplates}
			\end{table}
		\end{landscape}
		
		This template, document class, and guide aim to address these shortcomings by providing all the necessary information to create a well-structured thesis, along with examples to assist in formatting your thesis written in \LaTeX.
		To ensure the robustness and ease of maintenance, I developed the class file from the ground up keeping the additional required packages to a minimum.
		This makes this \LaTeX\ solution easier to maintain, update, and customize to suit different needs from different areas of the \University.
		A key goal with this was to reduce the traditionally steep learning curve associated with \LaTeX\ to ensure that anyone could create an outstanding thesis.
		
		While the class file (\path{ualberta.cls}) deserves its own comprehensive documentation, this document will focus on more specifically on the template file (\path{ualberta.tex}), as well as the following points:
		\begin{itemize}
			\item Installation and basic usage of \LaTeX.
			\item Document structure and formatting.
			\item Inclusion of figures and tables.
			\item Inclusion of plots and graphs.
			\item Handling mathematical equations.
			\item Citations and references using BibTeX.
			\item Use of JabRef---Reference Manager.
			\item Inclusion of Code and PDF's.
			\item And more.
		\end{itemize}

		This guide does not cover advanced \LaTeX\ programming or extensive customization of document classes.
		Instead, the class file \path{ualberta.cls} provides all the major document and formatting requirements as provided by \Fac, while this document offers references on how to include the various elements that might be required in a thesis.
		This includes all of the explanations of the packages and macros needed to perform the examples.

	\section{Organization of the Thesis}
		The thesis is organized into several chapters, each addressing a specific aspect of writing a thesis in \LaTeX. 
		The breakdown is as follows:
		\begin{itemize}
			\item \textbf{\Cref{ch:gettingstarted}:}  \nameref{ch:gettingstarted}
			\item \textbf{\Cref{ch:documentstructure}:}  \nameref{ch:documentstructure}
			\item \textbf{\Cref{ch:figureandtables}:}  \nameref{ch:figureandtables}
			\item \textbf{\Cref{ch:plotsandgraphs}:}  \nameref{ch:plotsandgraphs}
			\item \textbf{\Cref{ch:mathematicalequations}:}  \nameref{ch:mathematicalequations}
			\item \textbf{\Cref{ch:citref}:}  \nameref{ch:citref}
			\item \textbf{\Cref{ch:JabRef}:}  \nameref{ch:JabRef}
			%\item \textbf{\Cref{ch:}:}  \nameref{ch:}
		\end{itemize}

		Each chapter provides detailed information, examples, and recommendations to help you navigate the thesis writing process within the \LaTeX\ ecosystem.

	\section{Summary}
		This chapter introduced the background, objectives, scope, and organization of the thesis. 
		The subsequent chapters delve into specific topics, offering practical guidance and examples for mastering the art of writing a thesis in \LaTeX.
		Through this process, you will develop an understanding of how to create a thesis and manipulate content within the \LaTeX\ ecosystem to produce an exceptional document.