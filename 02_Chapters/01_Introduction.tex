\chapter{Introduction}\label{ch:Introduction}\section{Background}
	As a graduate student from the University of Alberta, I understand the daunting task that is associated with writing a Thesis that conforms to the guidelines outlined in the GPS Minimum Thesis Formatting Requirements. 
	It can also be very frustrating to write long, equation and figure heavy, document in a word processor that is prone to crashes, file corruption, seemingly random changes to the formatting, and that do not output a document in the required PDF/A format for submission to GPS.
	
	Due to these problems, lots of students attempt to use an alternative to traditional word processors: \LaTeX. 
	
	\LaTeX\ allows students/researchers to focus on either the writing of the document or the formatting. 
	Because the writing is separated from the formatting, the writing of the documents can be performed in much more lightweight text editors, or \TeX\ editors (that also allow for the compilation of the documents). 
	These editors are usually able to save the work after every keystroke and due to the plaintext nature, are not generally susceptible to file corruption.
	\LaTeX\ has the added benefit of providing a consistent and professional look and feel throughout the document. 

	\section{Objectives}
		The main objectives of this thesis are:
		\begin{enumerate}
			\item To provide a comprehensive guide on writing a thesis using \LaTeX.
			\item To assist students and researchers in mastering the nuances of \LaTeX\ document preparation.
			\item To showcase best practices for structuring and formatting a thesis in \LaTeX.
		\end{enumerate}

	\section{Scope and Limitations}
		While there are existing templates for writing a thesis for the University of Alberta in \LaTeX\ (see \Cref{tab:OtherTemplates} for a list of all other templates I was able to find), there does not appear to be a template for \LaTeX\ that provides students all the information required to write an outstanding thesis.
		
		\begin{landscape}
			\begin{table}[p]
				\centering
				\small
				\begin{tabularx}{\linewidth}{ccL}
					\toprule
						Developer & Last Updated & Link to Template Source\\
					\midrule
						Shivam Garg & May 29, 2023 & \url{https://github.com/svmgrg/ualberta_thesis_template}\\
						Henry Brausen & Feb 11, 2022 & \url{https://github.com/henrybrausen/thesis_template}\\
						Bernard Llanos & Oct 05, 2019 & \url{https://drive.google.com/file/d/1wKS8fu5e6qiVDRt0VUzEtlW8p7uMyz1T/view?usp=sharing}\\
						John Bowman & Sep 30, 2019 & \url{https://github.com/vectorgraphics/uofathesis}\\
						\makecell{Hongtao Yang\\\&\\Benjamin Bernard} & Sep 28, 2017 & \url{https://github.com/adrs0049/ThesisTemplate}\\
						\makecell{GAME\\\&\\Hongtao Yang} & Feb 03, 2016 & \url{https://www.ualberta.ca/computing-science/media-library/grad/candidacy-template-tex.tex}\\
						Steven Taschuk & Mar 21, 2012 & \url{https://github.com/stebulus/ualberta-math-stat-templates/tree/master/thesis}\\
						CMENG & Jul 19, 1999 & \url{https://sites.ualberta.ca/CMENG/research/new-control/stythes.html}\\
					\bottomrule
				\end{tabularx}
				\caption{List of Other Available Templates.}
				\label{tab:OtherTemplates}
			\end{table}
		\end{landscape}
		
		This template/document aims to defeat this shortcoming by providing all the necessary information to create a well structured thesis, as well as providing examples to assist in the formatting of documents written in \LaTeX{}.
		This thesis focuses on the following aspects:
		\begin{itemize}
			\item Installation and basic usage of \LaTeX.
			\item Document structure and formatting.
			\item Inclusion of figures and tables.
			\item Handling mathematical equations.
			\item Citations and references using BibTeX.
			\item Introduction to advanced topics and recommended packages.
		\end{itemize}

		However, it does not cover advanced \LaTeX\ programming or extensive customization of document classes.
		Mainly because I did the heavy lifting for you; the class file, \path{ualberta.cls}, provides all the major document requirements while this document provides the references of how to include all the bits and bobs that one might what in a thesis.

	\section{Organization of the Thesis}
		The thesis is organized into several chapters, each addressing a specific aspect of writing a thesis in \LaTeX. 
		The breakdown is as follows:
		\begin{itemize}
			\item \textbf{\Cref{ch:gettingstarted}:}  \nameref{ch:gettingstarted}
			\item \textbf{\Cref{ch:documentstructure}:}  \nameref{ch:documentstructure}
			\item \textbf{\Cref{ch:figureandtables}:}  \nameref{ch:figureandtables}
			\item \textbf{\Cref{ch:plotsandgraphs}:}  \nameref{ch:plotsandgraphs}
			\item \textbf{\Cref{ch:mathematicalequations}:}  \nameref{ch:mathematicalequations}
			\item \textbf{\Cref{ch:citref}:}  \nameref{ch:citref}
			\item \textbf{\Cref{ch:JabRef}:}  \nameref{ch:JabRef}
			%\item \textbf{\Cref{ch:}:}  \nameref{ch:}
		\end{itemize}

		Each chapter provides detailed information, examples, and recommendations to help you navigate the Thesis writing process and how that integrates within the \LaTeX\ ecosystem.

	\section{Summary}
		This chapter introduced the background, objectives, scope, and organization of the thesis. 
		The subsequent chapters delve into specific topics, providing practical guidance and examples for mastering the art of writing a thesis in \LaTeX.
		Further, you will develop an understanding of both how to create a Thesis, and how to manipulate the content within the \LaTeX\ ecosystem to create an incredible document.