\chapter{Getting Started with \LaTeX}
	\section{Installation}
		To begin using \LaTeX, you need to install a \LaTeX\ distribution on your computer. Here are the steps for installing \LaTeX\ on different platforms:

		\subsection{Windows}
			For Windows users, you can install \href{https://miktex.org/download}{MiKTeX} or \href{https://tug.org/texlive/windows.html}{TeX Live}. Download the installer from the respective websites and follow the installation instructions.

		\subsection{Mac}
			On Mac, you can use \href{https://tug.org/mactex/}{MacTeX} or \href{https://miktex.org/download}{MiKTeX}. Download the package from the respective websites and follow the installation instructions.

		\subsection{Linux}
			For Linux users, \href{https://tug.org/texlive/quickinstall.html}{TeX Live} is a common choice or one can use \href{https://miktex.org/download}{MiKTeX}. Use your package manager to install it, or download the installer from the respective websites.

	\section{Basic Document Structure}
		Once \LaTeX{} is installed, you can create a basic \LaTeX{} document. Here is a minimal example:

		\begin{lstlisting}[style=LaTeXStyle]
\documentclass{article}
\begin{document}
    \title{My First \LaTeX{} Document}
    \author{Your Name}
    \date{\today}
    
    \maketitle
    
    Hello, \LaTeX{}!
\end{document}
		\end{lstlisting}
		This example demonstrates a simple \LaTeX\ document with a title, author, and date. 
		The \lstinline|\maketitle| command generates the title information.

	\section{Other Software Considerations}
		While by itself, \LaTeX\ can be used with just a text editor and compiler, there are some additional software resources that will be very useful.
		
		First of these softwares is a Reference Manager.
		
		\note{While a reference manager is suited for creating a thesis in \LaTeX, it is not exclusive for use with \LaTeX. In fact, even if you are writing your thesis in word I would recommend the use of one of the suggested reference managers to help keep track of all your references, and to keep track of the information you found within your references.}
		
		Second of these (maybe should have been first or earlier in the document) is a different \LaTeX\ editor.
		While all of the distributions above come with TeXWorks (a simple editor and compiler) you might find it more beneficial to have some additional features.
		For this I recommend one of the following:
		\begin{itemize}
			\item \href{https://www.texstudio.org/}{TeXstudio}
			\item \href{http://www.xm1math.net/texmaker/}{Texmaker}
			\item \href{http://www.texniccenter.org/}{TeXnicCenter}
			\item \href{https://www.overleaf.com/}{Overleaf *See Note Below}
		\end{itemize}
		I am sure there are others, but any of these will do you fine for writing your thesis.
		
		\note{Due to changes with Overleaf, I recommend it less and less. While it is convenient as it is web based, they have restricted it more and more forcing the users to now pay more for the software... including to get longer compile times. While this is not an issue for short articles, in longer documents, such as a thesis, this can be a large hinderance and cost (\$100CAD/year with student discount).}